\pagestyle{fancy}
\begin{refsection}
\pagenumbering{arabic}
This first chapter presents and introduction and review of the main subjects which are relevant in the thesis. This review will help the reader to understand the vantguard of this subjects and how this thesis will contribute to knowledge in such areas. Initially an overview on active packaging and its most common application, Oxygen Scavengers (OS), is presented in Section \ref{sec:act_pac} and \ref{sec:OS}. Next, a bibliographic research over Poly-Unsaturated Fatty Acids (PUFAs) and Linseed oil oxidation kinetics is exposed in section \ref{sec:Linseed_Oxid.}. In section \ref{sec:modeling}, a review about active packaging modeling, focused on OS material is presented. Here, a classification of the different modeling techniques used in literature is proposed. Consequently, models were classified as homogeneous, heterogeneous or multi-film.

Lastly, section \ref{sec:objectives} presents the general and specific objectives of this thesis followed by the strategy proposed to fulfil them. 

\section{Active Packaging}\label{sec:act_pac}
 One of the biggest markets around the world is the packaging industry, which in 2019 had a price of \$ 917 billion dollars, and is expected to reach a price of \$1.05 trillion dollars by 2024  \cite{Smithers2019The2024}. Within this industry, food packaging, is a mayor business sector representing almost half of the market \cites{TeckKim2014GeneralSystem}{robertson2016food}. Throughout its history, food packaging has played rather a passive roll in food storage by surrounding and protecting the edible from the exterior conditions, so the product reaches its final consumer. This traditional packages have four primary functions: containment, protection, convenience and communication. The first function deals with the transportation of the product, without this the product would be lost in its way during distribution. Secondly protection, which as mentioned before deals with preserving the food quality by  serving as a barrier between the product and the outside environment conditions which tend to reduce edible shelf life. The third function, convenience, deals principally with the concept of apportionment (size and shape of the package) \cite{robertson2016food}. Finally communication function where the package gives information about the product within it. This last function tends to be crucial in the costumer decision in whether to buy or not the product. 
 
 Over the last decades, there has been an increment on the consumer demand on high-quality, minimally processed safe food; as well as, the demand of  food distributor for longer shelf life products \cite{Yildirim2018ActivePackaging}.  As a response, innovative packages are being created, main one being, the intelligent packages and the active packages.  The former ones refers to packages with an enhance communication function.  This packaging systems are capable of sensing and monitoring the conditions at which the food is within it while being capable of communicating this state or quality of the product to a human \cites{Yam2005IntelligentApplications}{Kruijf2002ActiveAspects}. On the other hand, Actives Packages (AP) are those which have an advance protecting function. This are define as packaging system which interacts with the food by changing its internal environment through the release or absorption of chemical substances so that the product shelf life an quality is extended \cites{brody2001active}{Yildirim2018ActivePackaging}.
 
 Active packages can be classified into releasing systems and scavenging systems, and the name given to the different types of AP are given by the action the package do rather than what the impact of this action have on food. Within the releasing systems the most commercially successful are $CO_2$ emitters, Ethanol emmiters and Flavor/Odor releasers \cites{robertson2016food}{PereiradeAbreu2012ActiveIndustry}. In the case of carbon dioxide emitters, $CO_2$  has an antimicrobial effect helping in food preservation. This  gas is released by using technology such as absorbent pads or moisture-mediated bicarbonate chemicals \cite{Emanuel2019FoodPerspective}. As well as $CO_2$ releasers, ethanol exhibit antimicrobial (AM) effects even at low concentrations, that is why ethanol vapor releasers has been developed. This emitters usually come in sachets which apart from containing alcohol also contains traces of vanilla or other flavors to mask the odor of ethanol \cite{robertson2016food}. The latter realeser mentioned, Flavor or Odor emitters are used when long term storage of food occurs. In this cases chemical interactions between the food and the package can stake place causing an off-flavor production. To overcome this problems flavor releaser can be used in direct or indirect contact with the food \cite{Ahmed2017AFoods}.
 
On the other hand, the most used types of AP are scavenger systems, among those, moisture regulators, ethylene, carbon dioxide and oxygen scavengers stand out. In the case of moisture regulators, as the name indicates, are active packages which absorbs water or humidity from the package head space to suppress microbial growth. This scavengers come in form of sachets films and pads which typically contain calcium oxide, silica gel among other moisture absorbers \cite{Gaikwad2019MoistureApplications}. 

Ethylene ($C_2H_4$) scavengers are used mainly in plant packages. This occurs because ethylene gas plays a significant role as a hormone which helps plants flowering, so there exit an interest in reducing the concentration of this specie in head space gas. This helps in  extending harvest durability so there is a reduction in  post-harvest lost \cite{Gaikwad2020EthyleneProduce}. With respect to $CO_2$ scavengers, these works by using either a physical absorbent (such as carbon powder) or a chemical one (like $Ca(OH)_2$ or $Mg(OH)_2$) to reduce the quantity of carbon dioxide present in the package. This can be a problem when the edible (such as roasted coffee) releases big amounts of $CO_2$ which can cause package swelling and even bursting \cite{robertson2016food}. 
Finally the most common type of AP are Oxygen Scavengers (OS) which were one of the first innovative packages to be invented on the decade of 1930's  \cite{Singh2011ActiveTrends}. 
 
\section{Oxygen Scavengers (OS)}\label{sec:OS}
There exist several causes of food spoilage including microbial growth,  chemical interaction within food components and enzymatic action present in the product \cite{Gaikwad2018OxygenPackaging}. The presence of oxygen within headspace gas is one of the principal causes of this food spoilage. The presence of this gas allows rapid oxidation of fats or vitamins while promoting aerobic microorganism proliferation. Oxygen may be present in food packages due to inefficient vacuums,  mixture of gases containing oxygen residues and permeation through the packages walls \cite{Souza2012OxygenPreservation}. With the purpose of reducing the quantity of oxygen in contact with oxygen-sensitive products, edibles tend to be packed under modified atmospheres (MAP) or under vaccum. This techniques can reduce residual oxygen concentration in head space between 0.5 vol\% to 2 vol\% which although it is a low percentage it may prove itself  to be destructive \cite{Gibis2011OxygenApplication}.   Therefore as away to improve food protection against oxygen, oxygen scavengers were invented and its nowadays one of the most important active packaging technology. An OS can be defined as a material in which a chemical  is incorporated into a package structure and may combine with $O_2$ to effectively remove oxygen from the inner package environment \cites{brody2001active}{Cooksey2010OxygenSystems}. Studies have found that OS are able to reduce $O_2$ concentration in head space to less than 0.1 vol\% helping prolonging shelf life \cite{Mills2006DemonstrationFilm}.
Given that in principle any oxidizing substrate can be used as an OS,  there exist several types of candidates for application in this technology. Different types of OS can be classified by its nature as metallic, organic, polymer based, enzymatic and natural oxygen scavengers \cites{Gaikwad2018OxygenPackaging}{Dey2019OxygenReview} as well as by the form they are implemented within the package. The market of OS is distributed by presentation according to figure \ref{fig:presentacio_os}. 
\begin{figure}[H]
        \centering
         \begin{tikzpicture}
         \pie[] {
            43.8/ Bottles,
            31.5/ Crowns and caps,
            22.7/ sachets,
            2/ Films
         }
        \end{tikzpicture}
        \caption{Distribution of the market by OS presentation \cite{Cooksey2010OxygenSystems}}
        \label{fig:presentacio_os}
\end{figure}
 A list of some of the most commercially used OS is shown in table  \ref{table:OS_prod}

\begin{table*}[ht]
\caption{Producers of oxygen scavengers, their name and active agent \cite{Vermeiren2003OxygenScavengers, Souza2012OxygenPreservation}}
\resizebox{\textwidth}{!}{%
\begin{tabular}{lccl}
\hline
Company & \begin{tabular}[c]{@{}c@{}}Commercial\\ Name\end{tabular} & Type & \begin{tabular}[c]{@{}c@{}}Active \\   Agent\end{tabular} \\ \hline
Mitsubishi Gas Chemical Co.,Ltd. (Japan) & Ageless & Sachets and & Iron based \\
 &  & Labels &  \\
Toppan Printing Co., Ltd. (Japan)& Fresilizer & Sachets & Iron Based \\
Toagosei   Chem. Ind. Co. (Japan) & Vitalon & Sachets & Iron Based \\
Nippon Soda Co., Ltd. (Japan) & Seaqul & Sachets & Iron Based \\
Finetec Co., Ltd. (Japan) & Sanso-cut & Sachets & Iron Based \\
Toyo Pulp Co. (Japan) & Tomatsu & Sachets & Catechol \\
Toyo Seikan Kaisha Ltd. (Japan) & Oxyguard &Plastic Trays & Iron Based \\
Dessicare Ltd. (US) & O-Buster & Sachets & Iron Based \\
Multisorb technologies Inc. (US) & FreshMax & Labels & Iron Based \\
 & FreshPax & Sachets & Iron Based \\
Amoco Chemicals (US) & Amosorb & Plastic Film & Unknown \\
Ciba Specialty chemicals (Switzerland) & Shelfplus O2 & Plastic Film & Iron Based \\
W.R. Grace and Co. (US) & PureSeal & Bottle crowns & Ascorbate/metallic salts \\
 & Darex & Bottle crowns, Bottle & Ascorbate/sulphite \\
CSIRO/Southcorp Packaging (Australia) & Zero2 & Plastic Film & Photosensitive dye/\\
 &  & & organic compound \\
Cryovac  Sealed Air Co. (US) & OS1000 & Plastic Film & Light activated scavenger \\
CMB Technologies (UK) & Oxbar & Plastic Bottle & Cobalt catalyst/ \\
 &  & & nylon polymer  \\
Standa Industrie (France) & ATCO & Sachets & Iron Based \\
 & Oxycap & Bottle crowns & Iron Based \\
 & ATCO & Lables & Iron Based \\
Bioka Ltd. (Finland) & Bioka & Sachets & Enzyme Based \\ \hline
\end{tabular}%
}
\label{table:OS_prod}
\end{table*}

\subsection{Metallic Scavengers}\label{subsec:metallic_os}
Within the different natures of the OS present in the market, metallic scavengers have the largest segment and have been used for many years. The majority of this scavengers come in forms of sachets containing iron powder or ferrous salts. The working principal of this technology is oxidation of 
metallic salts which are activated by the presence of water when exposed to air. The sachets are stable given that they do not react immediately when exposed to air \cite{SouzaCruz2005AbsorvedoresRevisao}. The reaction mechanism of iron as an OS is described by the following equations \cites{Souza2012OxygenPreservation}{Dey2019OxygenReview}{cruz2012oxygen}. 

\reaction{Fe -> Fe^{+2} +2e^-}
\reaction{$\frac{1}{2}$ O2 +H2O +2e^- ->2OH^-}
\reaction{Fe^2+ +2OH^- -> Fe(OH)_2}
\reaction{Fe(OH)_2 + $\frac{1}{4}$ O_2 + $\frac{1}{2}$H2O -> Fe(OH)_3}

The performance of the sachets containing iron salts were studied along with vacuum technology in packages of lasagna pasta. Results showed that aerobic microorganism growth was very slow and reached a maximum rate till 30 days after packing \cite{Cruz2006EvaluationPacked}. The commercial success of these type of scavengers comes from its efficiency, low price and capacity of oxidation \cite{Gaikwad2018OxygenPackaging}. To increment the scavenging capacity of iron powder, this was impregnated in polymeric films
, and experimental results in sausages packages showed that the scavenging capacity was 33 $cm^3 O_2/m^2 film$ in 4 days. This value of the scavenging capacity  was accounted to the oxygen permeability of the polymer \cite{Gibis2011OxygenApplication}. 

Other type of metallic OS is palladium, which was studied by Yildirim et al. \cite{Yildirim2015DevelopmentThickness} by deposing it into film layers together with MAP. They found and strong dependence between the scavenging rate and the palladium coating film thickness, determining that the optimum thickness value was  between 0.7nm  and 0.34 nm. Another important result found was that polyethylene terephthalate (PET), oriented polypropylene (PP) and  poly lactic acid (PLA) were better matrices for palladium based OS \cite{Gaikwad2018OxygenPackaging}. Palladium OS presents several  advantages over iron powder, given its low toxicity and high efficiency.

Some of the main drawbacks of metallic scavengers are potential contamination of the edible with metal and the strong dependence of the scavenger activity on humidity (which is why it is not suitable for dry applications) and on temperature. Another possible problem is activating inadvertently metal detectors as well as inhibiting hitting in microwave ovens.   

\subsection{Organic Scavengers}\label{subsec:organic_os}
Given the previous limitations that metallic scavenger have, the  use of organic-based OS has increased in the last few years \cite{Gaikwad2018OxygenPackaging}. Organic scavengers can go from small oligomers added to films,  to  oxygen scavenging polymers  with side-chains that react with oxygen. One of the most used organic scavengers is ascorbic acid, whose reaction mechanism is based on the oxidation of this specie to dehydroascorbic acid. This reaction tends to be slow, its rate depends on the concentration of oxygen  present which is why its scavenging capacity is enhanced with catalyst like cooper or iron. The effect of this metal catalyst  increases the scavenging capacity of ascorbic acid by 16.4\% \cite{Uluata2015HowEmulsions}. As in the case of metallic scavengers, ascorbic acid's scavenging rate increases in presence of water or relative humidity. When applied in starch films, this type of technology is able to capture 13.5 $ml O_2/g film$  in 15 days \cite{Mahieu2015ThermoplasticContent}.  

Another organic scavenger is $\alpha$-tocopherol, which has become a promising option for applications OS technology given that it is natural free radical scavenger and its biodegradable. To avoid the quick degradation of this substance it is usually encapsulated in PLA micro-particles or loaded in PLA films \cites{DiMaio2014PreparationPackaging}{Scarfato2017PreparationApplications}. 
Organic-based scavenger present several drawbacks such as greater expenses and lower scavenging capacity when applied to a polymeric matrix. However the use of this type of substances in OS technology is promising, given that it allows a more sustainable way of producing AP packages (like for example the development of edible OS films \cite{Janjarasskul2011WheyAcid}) \cite{Gaikwad2018OxygenPackaging}. 

\subsection{Polymer-Based Scavengers}\label{subsec:polymer_os}
As mentioned in section \ref{subsec:organic_os},  a common practice when using organic scavengers is using a polymer which is suitable for packing purposes and modified with reactive side chains attached to the substance backbone. This can be  achieved by making homogeneous blend of reactive substances with the polymeric matrix \cite{Cooksey2010OxygenSystems}. Some of the most common polymers used for OS purpose are polyamides and polyolefins which can be incorporated into the backbone of a base polymer or can be used as an OS layer in a multilayer package arrange \cite{Gaikwad2018OxygenPackaging}. An example of this type of polymers is 1,4-polybutadiene. The autoxidation of polymers needs an agent which triggers these reaction, that is why usually when using reactive polymers a photoinitiator, metal catalyst or  UV-light is used \cite{Galdi2008ProductionActivity}. In the case of 1,4-polybutadiene, a cobalt catalist is used to increase its scavenging rate. The reaction mechanism by which this reaction occurs is shown in the following equations \cite{Li2012Characterization14-Polybutadiene}.

\reaction{O2 +RH -> ROOH}
\reaction{Co^+2 + ROOH -> Co^3+ +OH- +RO}
\reaction{Co^+3 + ROOH -> Co^2+ +OH- +ROO^-}

where RH refers to the allylic carbon-hydrogen bonds of the polymer which undergoes oxidation. Within the disadvantages of using this OS technology 
is that as oxidation of the polymer occurs there is an increases in concentration of byproduct such as  organic acids, aldehydes or ketones  which can affect the quality of the edible by developing odour or rancid off-flavor which affects the taste of the product \cites{Li2012Characterization14-Polybutadiene}{Gaikwad2018OxygenPackaging}. To reduce the quantity of by product produced by oxidation Miranda  and Speer \cite{Miranda2005PolylacticArticle}  suggested  the addition of a barrier layer to impede the migration of this substances. 

\subsection{Polyunsaturated Fatty Acids scavengers}\label{subsec:PUFA_os}
As an alternative to the different types of OS mentioned before in this document, the use of polyunsaturated fatty acids (PUFAs) is an excellent candidate for OS technology. Most of the scavengers presented in previous sections presented the disadvantage that for an optimal performance they needed the presence of water within the package head space, causing this OS to be unsuitable in application for dry food. The quality of this class of edible  declines rapdly when applying the usual OS technology given the migration of water from the oxygen scavenging system into the food. The operation and functioning of PUFAs as OS does not depends on the quantity of water present in the package which makes it an ideal candidate for AP for dry food. Based in this principle, Mitsubishi Gas Chemical  Co. established a patent over PUFAs as a reactive  agent.  They established that this acids (i.e. oleic, linoleic or linolenic acid) can be contained within a carrier oil such as sesame or soybean oil which in turn is compounded with a transition metal catalyst and a carrier substance like calcium carbonate. This latter substance is responsible for the solidification of the OS, which is latter made into granule or powder to use this component in sachets \cites{Vermeiren2003OxygenScavengers,cruz2012oxygen,Floros1997ActiveApplications}.

OS implemented in sachets have several problems, within which stands out the accidental release of the active OS component which could contaminate the food product and be ingested as well. Also sachets are limited to solid food packages disabling its application in liquid food \cite{Rooney2005IntroductionTechnologies}. To solve this limitations of the sachet form of OS, incorporation of this type of scavengers in packaging films is proposed  by  the Materials and
Manufacturing research group (CIPP-CIPEM) of \textit{Universidad de los Andes}.  This group have developed in the last few years an OS based in linseed oil as active agent which is encapsulated within silica-gel micro capsules and incorporated within PP matrices. Experimental results have shown a good  performance of this type of films being able to scavenge 0.7 $mmol O_2 /g \hspace{5pt} linseed\hspace{4pt} oil$ in a 6 hour time span \cite{GarciaMora2015KineticScavengers}. 

\section{PUFA's and Linseed Oil Autoxidation Kinetics}\label{sec:Linseed_Oxid.}
Given the applicability of PUFAs as oxygen scavengers it is important to study the autoxidation kinetics of these type of substances. Understanding this mechanism its also important for the study of degradation of lipid containing food given that this process causes sensory and nutritional degradation. This is because most fats come in form of triglycerides which in turn are made of glycerol and a combination of saturated and unsaturated fatty acids which link with glycerol by its hydroxyl groups \cite{Labuza1971KineticsFoods}. The main unsaturated fatty acids present in the triglicerydes are oleic, linoleic and $\alpha-$linolenic acid whose structure are presented in Figure \ref{fig:estructura_pufa}. 


\begin{figure}[!ht]
\centering
\setchemfig{atom sep=2em}
\subfloat[]{
   \chemfig{HO-[:30](=[2]O)-[:-30]-[:30]-[:-30]-[:30]-[:-30]-[:30]-[:-30]-[:30]=_-[:-30]-[:30]-[:-30]-[:30]-[:-30]-[:30]-[:-30]-[:30]}\par
}

\subfloat[]{
  \chemfig{HO-[:30](=[2]O)-[:-30]-[:30]-[:-30]-[:30]-[:-30]-[:30]-[:-30]-[:30]=_-[:-30]-[:30]=_-[:-30]-[:30]-[:-30]-[:30]-[:-30]}\par
}

\subfloat[]{
  \chemfig{HO-[:30](=[2]O)-[:-30]-[:30]-[:-30]-[:30]-[:-30]-[:30]-[:-30]-[:30]=_-[:-30]-[:30]=_-[:-30]-[:30]=_-[:-30]-[:30]}\par
}
\caption{Structure of the principal unsaturated fatty acids where (a) corresponds to oleic acid, (b) to linoleic acid and (c) to $\alpha$-linolenic acid}
\label{fig:estructura_pufa}
\end{figure}

This unsaturated fatty acids are very susceptible to be attack by oxygen due to the presence of the double bonds in their structure (Figure \ref{fig:estructura_pufa}) and as a product of this reaction rancid flavors in food develops. Labuza \cite{Labuza1971KineticsFoods} was one of the first one to study the kinetics of autoxidation,  indicating that the reaction mechanism by which oxidation occurs is based in the presence of free radicals which react with external oxygen and causes the production of peroxides and hydroperoxides. This latter substances are unstable so they break down again to producing free radicals, thus initiating a chain reaction.  As a consequence of the chemical mechanism described,  the following reactions take place:
\begin{center}
    \textbf{Initiation }
    \reaction{$\text{Initiator}$ ->[k_i] $\text{free radicals}$ }
    \textbf{Propagation }
    \reaction{R^. +O2 ->[k_o] ROO^. }
    \reaction{ROO^. +RH ->[k_p] ROOH + R^. }
    \textbf{Termination}
    \reaction{ROO^. + ROO^. ->[k_{t1}] Non-Radical Products}
    \reaction[react:termt2]{ROO^. + R^. ->[k_{t2}] Non-Radical Products}
    \reaction[react:termt3]{R^. +R^. ->[k_{t3}] Non-Radical Products}
\end{center}

Where \ce{R^.} is the substrate free radical, \ce{ROO^.} are peroxy radicals and $k_i$, $k_o$ , $k_p$ and $k_{ti}$ are the initiation, oxygen step, propagation step and termination rate constants respectively. Based on the chemical reactions presented before, oxidation reaction as well as peroxide reaction kinetics will be given by equation \ref{eq:cinetica_labuza}

\begin{equation}
    -\frac{dO_2}{dt}=\frac{d(ROOH)}{dt}=\frac{k_pR_i^{0.5}}{2k_{t1}^{0.5}} [RH] \frac{[O_2]}{[O_2]+ \left(\frac{k_{t3}}{k_{t1}}\right)^{0.5}\frac{k_p}{k_o}[RH]}
    \label{eq:cinetica_labuza}
\end{equation}
Where $R_i$ corresponds to the initiation rate. In this case Labuza suggest that initiation requires that a direct attack of oxygen to UFAs has to occurs such that the formation of few peroxides happens (i.e. \ce{RH + O2 -> $\text{Free Radicals or }$ ROOH}). Once there is enough $ROOH$ present the chain reaction can take place by two initiation mechanism,  monomolecular decomposition (\ce{ROOH -> RO^. + ^. OH}) or bimolecular decomposition (\ce{2ROOH -> ROO^. +RO^. +H2O}). 

Given the complexity of the kinetics expression presented earlier, in 1995 Adachi et al. \cite{Adachi1995AutoxidationEsters} proposed a simple kinetic expression to describe the entire autoxidative process of PUFAs. To do this, they based in the work of Bolland \cite{Bolland1949KineticsOxidation} who proposed that the rate of reaction for the autoxidation of ethyl linoleate is 

\begin{equation}
    rate=k_\alpha \frac{[O_2]}{K+[O_2]}[RH][ROOH]
\end{equation}

where $k_\alpha$ is the rate constant and $K$ is a saturation constant. This expression was modified assuming that $[ROOH]$ is proportional to the rate of consumption of substrate, so with this idea the rate of reaction is going to be given by 

\begin{equation}
    \frac{dY}{dt}=-\frac{k_x[O_2]}{K+[O_2]}Y(1-Y)
    \label{eq:rate_adachi}
\end{equation}

where $Y$ represents the fraction of unoxidised substrate. The results obtained when comparing the kinetics for Ethyl $\alpha$-linolenate and Ethyl linoleate showed that the expression \ref{eq:rate_adachi} could predict the entire substrate concentration profile for the latter but only predicted the early stages of autoxidation for the former ($Y<0.5$). This investigation was continued by  Ishido et al. \cite{Ishido2001OxidationEster} using the same model presented in \cite{Adachi1995AutoxidationEsters} but using it for combinations of fatty acids (3-n and 6-n PUFAs). They found that in this case the constant $k_x$ was proportional to the molar fraction of substrate present indicating that when mixed unsaturated and saturated fatty acids, this last specie acts merely as a diluent.  
Continuing with the study of the kinetic rate of oxidation  of unsaturated fatty acids, Richaud et al. \cite{Richaud2012RateChemiluminescence} studied the mechanism of oxidation of unsaturated fatty esters which is similar to the PUFA one. He based his work in the kinetics developed by Audini et al. \cite{audouin1995close}. This kinetics differs form the one proposed by  Labuza et al. \cite{Labuza1971KineticsFoods}, in the bi-molecular initiation and ignoring the  contribution of the termination reactions involving \ce{R^.} radicals (Reactions \ref{react:termt2}
 and \ref{react:termt3}). This last assumption was made considering that the reaction takes places under excess of oxygen reason why oxygen propagation reaction will occur very fast increasing $[ROOH]$ concentration and causing that only termination associated with hydroperoxide is considered. With this reaction mechanism the oxidation kinetics is described by the following equations 
 
  \begin{gather}
     \frac{d[\ce{R^.}]}{dt}=k_1[ROOH]^2-k_o[\ce{R^.}][O_2]+k_p[\ce{ROO^.}][RH]\\
     \frac{d[\ce{ROO^.}]}{dt}=k_1[ROOH]^2+k_o[\ce{R^.}][O_2]-k_p[\ce{ROO^.}][RH]-2k_{t3}[\ce{ROO^.}]^2\\
     \frac{d[\ce{ROOH}]}{dt}=-2k_1[ROOH]^2+k_p[\ce{ROO^.}][RH]\\
     \frac{d[\ce{RH}]}{dt}=-k_p[\ce{ROO^.}][RH]
 \end{gather}
 With the previous system of equations shown before, Richaud is able to predict the initiation time (which is the time so there is enough radical concentration for the oxidation to begin) as well as the behavior of the species throughout the reaction. 
 
 Based in the kinetics and the results of Richaud et al., Garcia \cite{GarciaMora2015KineticScavengers} adjusted the velocity rate constant to predict the oxidation behavior of linseed oil. Linseed oil (Linum usitatissimum) is described as a natural dry oil (which means that it becomes solid when exposed to air \cite{Turner-Walker2012TheDegreasing}) which is mainly composed by  Linolenic acid (48-60 wt\%) , Linoleic acid (14-19 wt\%), Oleic acid (14-24 wt\%) , Stearic acid (3-6 wt\%) and Palmitic acid (6-7 wt\%) \cite{lazzari1999drying}. Given the high composition of Linolenic acid  within the oil, Garcia used the same  kinetic rate constants $k_2$ and $k_3$ found by Richaud for the corresponding methyl ester, and adjusted the velocities for the initiation and propagation reaction using experimental data. She also took into account the situation where $O_2$ was not in excess (contrary to what Richaud assumed in his model),  by taking into account termination reactions \ref{react:termt2} and \ref{react:termt3}. By considering limited $O_2$ concentration, the model was able to predict oxidative TGA results in air atmosphere (21 vol\%).
 
\section{Active Packaging Mathematical modeling}\label{sec:modeling}
In the next section a review over the different development in mathematical modeling on oxygen scavegers is going to be made. To do so, a classification of the trends regarding the different approaches of the papers will be done, those categories will be based on the empirical and theoretical nature of the models, more over the further will be classified depending on how the OS are implemented in the AP, and finally an overview on the models which show how external conditions (such as temperature and humidity) affect OS performance will be treated.

\subsection{Empirical Models}

Sometimes to develop a model, the theoretical approach may present certain  complications such as the solution of several differential equations which may be difficult to solve, even numerically. Taking this into account, researchers may develop empirical models which tend to be equations whose form its adjusted numerically form experimental data, so that a certain trend may be predicted. This method have the advantage that tend to be simple equations which are easy to use and can predict with good accuracy the behavior of the oxygen concentration within the package. A first approach was made by \textit{Larsen et al.} \cite{Larsen2002PredictingMethod} in 2002, where oxygen concentration was studied in nitrogen flushed packages for edibles. The author looked to predict oxygen behavior through time inside the package through a linear and exponential model. The linear model had as slope the oxygen transmission rate (OTR) which was assumed to be constant. Experiments showed that the linear model presented  good agreement at low OTR, but exponential model whose exponential constant was adjusted, showed a better agreement with oxygen concentration behavior for all OTR rates, having a higher degree of precision \cite{Larsen2002PredictingMethod}. In that same year \textit{Tewari et al.} \cite{Tewari2002AbsorptionScavengers} carried out an study on the absorption kinetics of several commercial oxygen scavengers (Ageless\textsuperscript{\tiny\textregistered} FX-100, FreshPax\textsuperscript{\tiny\textregistered} M-100 [Iron based] and Bioka\textsuperscript{\tiny\textregistered} S-75 [Enzime based]) which where implemented within polyester bags emptied of air. The oxygen concentration within the bag was monitored during several hours. They determined that oxygen scavenger rate of reaction where first order for all the commercial products. With that in mind, Tewari was able to adjust the rate of reaction constant to an Arrhenius Equation where the exponential and prexponetial factor where determined.

Next, considering the previous results of Larsen and Tewari, \textit{Van Bree et al.} \cite{VanBree2010PredictingTool} developed a computational tool which allowed to predict the head space oxygen level of multilayer polymer packing materials. The authors took into account multiple multilayer packing configurations which could or not implement oxygen scavengers inside the package (not within it). Furthermore, oxygen consumption by the food packaged was considered. In this study several polymers where modeled by adjusting the oxygen permeability of each material to an exponential equation in agreement with the work of Larsen. They found that for multiple layers permeability could be modeled empirically by equation \ref{eq:1}.

\begin{equation}
    P_T=\frac{L_T}{\frac{L_1}{P_1}+...+\frac{L_n}{P_n}}
    \label{eq:1}
\end{equation}

Where $L_T$ is the thickness of the total layer, while $L_n$ are the individual thicknesses of the layers and $P_n$  their individuals permeabilities. With this approach an overall 7.25\% of error between experimental and predicted data was obtained. Part of this error can be attributed to  the assumption that the kinetic rate of consumption of oxygen from the OS is of first order. This is based on the assumption that the kinetics of the reaction only depends on the oxygen concentration or the OS concentration which is a good approximation when $O_2$ or OS concentration is high (21\%). But for packages with modified atmosphere where OS is implemented within the package, this is not the case. Taking this into account, \textit{Pant et al.} \cite{Pant2018KineticScavenger} as well as \textit{Dombre et al.} \cite{Dombre2015ProtectionSolution} made  new studies on the kinetic modeling of OS, modeling the oxidation reaction as a second order reaction where oxygen and OS have a partial first order reaction each. In both cases the OS was implemented within a polyethylene (PET) matrix with the difference that Pascal did the study on Gallic acid OS while Dombre did not specified the chemical nature of its scavenger. The results obtained showed an RMSE of 0.0053 on the prediction of $O_2$ concentration proving that second order reaction is better to describe this type of situations. 

\subsection{Theoretical Models}

Despite the good agreement between the empirical predictions and experimental data, this type of models tend to be very specific on the OS used and its application. Theoretical models have the advantage that they explain different phenomena based on the same principals of mass transfer on the system as well as oxidations kinetics. The research on theoretical description of oxygen scavengers was found to center on the implementation of this in polymeric films.  Keeping this in mind, theoretical models will be classified in three groups depending on the how the OS is incorporated on the film: Homogeneous single layer Film, embedded film and homogeneous multilayer films. 

\subsubsection{Homogeneous Film}

The homogeneous films is one of the first approaches in modeling OS in films. In this models,  OS places within the film that react (active sites) are considered to be distributed homogeneously thought the film and all sites are assume to be equally big and reactive. Therefore the film is taken as if it was continuously reactive. In 2001 \textit{Yang and Cussler} \cite{Yang2001OxygenChemistry}  made a first attempt to model oxygen barriers which were reactive using films of ethyl cellulose with addition of linoleic acid a OS. Yang stated that even though the reactive groups implemented in the film could not stop oxygen to permeates the film, it could increase the time in which this occurs. For his study he used linoleic acid as the reactive component and ethyl-cellulose as the polymeric matrix. The model  used  was only valid for very fast reaction given that the part of the reaction in which Yang is centered is in describing the lag time (the time in which the oxygen permeates all the membrane). Yang found with his approach that the presence of scavenger even in a very permeable membrane, incremented the lag time in about 2000 times. Moreover, he determine the existence of a critical value of reagent concentration and film thickness under which oxygen concentration will not be high enough to activate the membrane enabling oxygen leaking.

In an attempt to generalize the results obtained by Yang, \textit{Carranza et al.} \cite{carranza2010modeling}  in 2010 modeled the oxygen concentration on the film through all stages in time and by not assuming that the reaction was  occurring much more faster than the diffusion, this enables the model to capture the effectiveness of the film  for reducing the oxygen headspace concentration. Theory showed that there existed three distinct regimes in the film's life: the early times where a flux plateau is observed, an intermediate time where a moving reacting front is observed and long times where most reactive times are consumed and lag time can be calculated. The model developed by Carranza had two dimensionless parameter $\Phi_H$ the Thiele modulus, and $\nu$ the ratio between oxygen content in the film and the scavenging capacity. The model presented accurate predictions specially on the flux plateau and the lag time.

The previous studies did not took into account the multiple reactions that where occurring beside oxidation of OS, which also affects the performance of the scavenger. In 2015, Garcia \cite{GarciaMora2015KineticScavengers}  did a kinetic modeling of OS taking into account a more profound study on the kinetics and chemical equations that occurred on the film. Her study was based on linseed oil as the active component of the OS which was incorporated into Polypropylene (PP) films in silica micro-capsules. The author identified  three reaction stages which were taken into account in modeling: Initiation in which  bi-molecular decomposition of the OS was considered,  a classical propagation scheme $O_2$ reacts with the decomposed OS and a termination scheme where several reaction occurs such that the OS becomes inactive. The kinetic parameters of the reactions in the three stages were determined experimentally and then they were implemented on the model, where only diffusion of the oxygen and reaction was taken into account. Simulation showed that for PP films, oxidation of the OS was not limited by diffusion but that the fact that Linseed oil was within micro-capsules inside the film slowed down the global oxidation rate, this given there was a low quantity of the oil in the system. When concentration of active compound increases as well as the thickness of the film, diffusion becomes the limiting factor in the oxidation of the OS.

\section{Objectives}\label{sec:objectives}

\subsection{General Objective}
\begin{itemize}
    \item To develop a computational design tool which predicts Linseed oil oxygen scavengers composite polypropylene film performance as function of different design variables and in presence of a competitive food agent for food packing applications.
\end{itemize}

\subsection{Specific Objectives}
\begin{itemize}
    \item To develop a mathematical model of the the films oxygen absorption performance by means of the homogeneous and composite film approach.
    \item To determine from different numerical methodologies for the resolution of stiff PDE which one suites the best (less computational time and better stability) to the solution of the mathematical model. 
    \item To determine and incorporate the complete oxidation kinetics of linseed oil oxidation within the physical model. 
    \item To study and implement several food oxidation kinetics into the model develop so that competition between OS film and eatable oxidation is predicted.
\end{itemize}
\vspace{\fill}



\printbibliography
\end{refsection}