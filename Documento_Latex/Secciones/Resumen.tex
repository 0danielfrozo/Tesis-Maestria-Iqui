\selectlanguage{spanish}
\begin{abstract}
\thispagestyle{plain}
\pagenumbering{roman} %
\setcounter{page}{3}
Uno de los principales tipos de empaques activos son los captadores de oxígeno (OS), los cuales son sustancias químicas cuyo propósito es reducir la cantidad de oxígeno residual en un paquete. Esto permite reducir el deterioro de los alimentos debido a la oxidación, prolongando así su vida útil. En el siguiente documento, el objetivo principal será desarrollar una herramienta computacional que permita el diseño de películas poliméricas captadoras de oxígeno. Actualmente, el grupo de materiales y manufactura CIPP-CIPEM de la Universidad de los Andes ha desarrollado películas de polipropileno que absorben oxígeno mediante la incorporación de aceite de linaza microencapsulado en sílice. En este caso, el aceite de linaza actúa como un agente absorbente de oxígeno activo. Para llevar a cabo la herramienta de diseño, surge la necesidad de desarrollar un modelo matemático que considere los detalles reactivos de la cinética de oxidación de la linaza, así como la distribución y evolución de los sitios activos dentro de la película polimérica. Con esto en mente, se desarrolló el enfoque de película reactiva y el modelo de película multicapa. Con el fin de implementar la naturaleza química de la oxidación del aceite de linaza, se realizó un ajuste sobre las constantes de velocidad cinética, así como sobre la concentración inicial de hidroperóxido y sustrato en el aceite. Los valores obtenidos logran predecir los perfiles de concentración experimentales, pero carecen de precisión para predecir los tiempos en que se desarrollan estos perfiles.
Con respecto a los métodos utilizados para resolver las ecuaciones del modelo matemático, se estudiaron varios algoritmos de diferencias finitas. Los resultados obtenidos al evaluar la precisión y el tiempo computacional mostraron que el mejor método para resolver el sistema de ecuaciones era el método de paso fraccional con una configuración de paso de tiempo adaptativo. Como conclusión principal de los resultados anteriores obtenidos, fue posible hacer una primera versión de la herramienta de diseño computacional. Esta versión es capaz de predecir la dinámica de los perfiles de reacción y concentración para películas absorbentes de oxígeno monocapa y multicapa homogéneas. La realización de este modelo implica un nuevo desarrollo en el diseño y modelado de películas poliméricas absorbentes de oxígeno con agentes orgánicos activos.



\end{abstract}
\selectlanguage{english} 
\begin{abstract}
\thispagestyle{plain}
\pagenumbering{roman} %
\setcounter{page}{4}
One of the main types of active packaging is oxygen scavengers (OS), which are chemical substances whose purpose is to reduce the amount of residual oxygen in a package. This allows reducing the food deterioration occurring due to oxidation, thus prolonging its useful life. In the following document, the main objective will be to develop a computational tool that allows the design of polymeric oxygen scavenger films. Currently, the CIPP-CIPEM materials and manufacturing group of the Universidad de Los Andes has developed polypropylene films which absorb oxygen through the incorporation of microencapsulated linseed oil in silica. In this case, linseed oil acts as an active oxygen absorbing agent. To carry out the design tool, it is necessary to develop a mathematical model that considers the reactive details of the oxidation kinetics of flaxseed as well as the distribution and evolution of the active sites within the polymeric film. With this in mind, the reactive film approach and the multilayer film model were developed.  To implement the chemical nature of linseed oil oxidation, an adjustment over the kinetic velocity constants as well as over the initial concentration of hydroperoxide and substrate was made. The values obtained achieve to predict the experimental concentration profiles but lack accuracy in predicting the times in which these profiles develop.  
Concerning the methods used for solving the mathematical model equations, several finite-difference algorithms were studied. The results obtained when evaluating the accuracy and computational time showed that the best method to solve the system of equations was the fractional step method with an adaptive time step configuration. As the main conclusion of the previous results obtained, it was possible to make the first version of the computational design tools. This version can predict the dynamics of the reaction and concentration profiles for homogeneous mono-layer and multilayer oxygen absorbent films. Carrying out this model implies a new development in the design and modeling of polymeric oxygen-absorbing films with organic active agents

\end{abstract}
\pagebreak

