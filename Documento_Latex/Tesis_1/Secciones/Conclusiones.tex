 \section{Conclusions}
The main purpose of an active package is to protect as long as possible the quality an integrity of a product by chemically and physically interacting with it and its surrounding until it reaches the final consumer. This technology been playing a fundamental role in the extension of the shell life of product in the last years. This is the reason why it is a field with a constant investigation growth. Despite the fact that there are different types of active packaging, the packaging time mostly used the type of packaging mostly used commercially are the oxygen scavengers. 

There exist several types of oxygen scavengers which can be used for  diverse kind of food depending on its specific characteristics. In the last few years the Material and Manufacturing investigation group  (CIPP-CIPEM) of \textit{Universidad de los Andes} has been developing a new kind of OS polypropylene film whose active component is linseed oil. This type of OS technology presents a promising in the oxygen scavenging market given its organic nature and the great oxygen uptake capacity of linseed oil. To study the performance of this new OS film,  experimental techniques have been used, but this tend to take several days and are of destructive. As an alternative to the experimental approach, the use of mathematical models and computational tools have been proposed.
 
A a first conclusion of this work is that a first version of a computational design tool for the OS film has been developed. This first version enables a user to study the performance of this new AP technology either present in a multilayer film or a solely film. The user is able to change several parameters of the system studied such as OS load, temperature, material of and dimension of the film among others. The results obtained are displayed in dynamic plots which enable a better visibility of the result profiles, and can be also saved in Excel files for a deeper analysis.

To develop this tool, knowledge about the oxidation kinetics of linseed oil is needed. As a second conclusion, the adjustment of the kinetic model was made by means of the Gauss-Newton technique and this was implemented into the reactive film model. New values of initial concentrations of hydroperoxyde and substrate were obtained, and using the kinetics velocities determined in literature. The new values allow to determine the order of magnitude of the concentration absorption but fail in predicting correctly the times in which reaction such as hydroxide decomposition occurs. 

Finally, in the seek  of implementing a tool which is computationally fast an accurate several numerical techniques were studied. The results showed that the system of PDEs which describes the dynamics of oxidation of the film are stiff, which means that implicit methods are needed to solve them efficiently. As a third conclusion, the fractional step method (FSM) modified with a variable time step was one that presented better results regarding efficiency and accuracy when solving the films dynamics which is why it was selected for the implementation of the computational design tool.
 
\section{Future Work}
There is much more work to be done in the path of developing a complete  computational design tool for the OS film developed by CIPP-CIPEM group. In first place the development of the \textbf{Analysis mode} of the tool must be developed. This mode will enable the user to study directly in the program the effect of varying design parameters over the oxygen scavenging performance of the film. The instruction or ''About this tool" sub-window must also be finished. Given the difficulty on developing an enriched text in Tkinter there exist the possibility of developing the instructions window in another  programming language and couple it latter to the actual interface.  

With respect to the kinetic model adjustment, a deeper study must be made using experimental data obtained in the oxidative TGA technique. this data will enable to make more trustable adjustment and in turn correct the problem of inaccuracy in time prediction. 

Likewise, the implementation of the heterogeneous film model is to be done. This model although it is more complex, describes more accurately the physics of the OS film. This is because the film is not reactive itself but contains spherical actives sites which are reactive. In that sense it is expected that the heterogeneous film model has better results predicting the OS film performance. Finally another addition to the model must be made, regarding oxidation competition between the AP and the edible inside it. Once implemented, the design tool is going to be able to indicate how much longer is a food  shelf life given the OS film in its package. 