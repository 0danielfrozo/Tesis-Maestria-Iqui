\begin{refsection}
In this chapter, the methodology followed for the development of this thesis is going to be carried out. This section is going to be divided in two main parts. The first one, will describe the experimental procedures carried out including the linseed oil oxidation kinetic determination, the silica micro-capsules synthesis, the OS film elaboration and the head-space oxygen concentration evolution test. The second part will refer to the computational procedure followed for the development of the design tool program, this includes the physical model description, the numerical resolution approach, the kinetic parameters adjustment strategy and finally the graphical user interface (GUI) programming. 

\section{Experimental Procedure}
\subsection{Silica micro-capsules synthesis}
Regarding the microcapsule synthesis, this has been carried out by the CIPP-CIPEM using the sol-gel emulsion templating methodology. This procedure consists of preparing an emulsion in which the disperse phase is the chemical specie that is desired to be encapsulated. In this case, the surfactant or emulsifier will serve as a template where tetraethyl orthosilicate (TEOS) precursor undergoes hydrolysis and condensation reactions to form the silica polymeric microcapsule \cites{GonzalezPungo2018EvaluacionActivos, GomezAlfonzo2018AceiteOxigeno} To obtain 2.34g of dry product, initially 85.5 mL of water, 0.48g of cetrimonium bromide (CTAB) and 36mL of ethanol are mixed in a 250mL beaker at a speed of 700rpm during 5 minutes. Afterward, 1.5 mL of double-cooked linseed oil is added drop by drop to the previous mix, and immediately after, 6mL of TEOS is slowly added to the beaker, which is left mixing during 5 minutes more. Finally, 3mL of 25\% ammonia solution is added to the beaker to catalyze the reaction, which is left mixing at 700 rpm for 24 hours. The mix obtained is centrifuged at a velocity of 4000 rpm for ten minutes as a result the micro-capsules obtained are separated from the liquid phase of the mix. Next, the solid is dried in a vacuum oven at 70\degree C and a pressure of $-0.06 MPa$ for 4 hours. After that time, the microcapsules are ready to be incorporated into the polymeric film. Next, an image of the product obtained is shown in figure \ref{fig:capsules}. 










\end{refsection}